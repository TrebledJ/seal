\subsection{Tuples}

\begin{lstlisting}
object Tuples {
  def foo(v: (Int, Boolean)): (Int, Boolean) = { // Type.
      v match {
          case (i, false) $\Arr$ // Pattern matching.
              (i, false) // Literal.
          case (i, true) $\Arr$
              (-i, false)
      }
  }

  val a: (Int, Boolean) = foo((1, true));
  val b: (Int, Boolean) = foo((-2, false));
  val a0: Int = a(0); // Variable access.

  Std.printString("a: (" 
    ++ Std.intToString(a0) ++ ", " 
    ++ Std.booleanToString(a(1)) 
    ++ ")"); // a: (-1, false)
  
  // Error: only int literals allowed as tuple accessors.
  // val a1: Int = a(0 + 1); 

  // Error: can't access index out of range.
  // val a1: Int = a(2); 
  
  val c: String = ("hello", "there")(0); // Literal access.
  Std.printString(c)
}
\end{lstlisting}
See \texttt{extension-examples/Tuples.scala} for a fuller demonstration.

There are three primary tuple operations:

\begin{enumerate}
  \item Construction: \code{(x, y)}
  \item Access: \code{t(i)} (for integer literal \code{i})
  \item Pattern matching: \\
    \code{t match \{ case (x, y) $\Arr$ ... \}}
\end{enumerate}

\noindent
Tuples were introduced both on the value level and on the type level, and have the usual one-to-one correspondence between a field's type and value. For example, \code{("a", 1)} has type \code{(String, Int)}.

Further semantic rules are as follows:

\begin{itemize}
  \item Tuples may hold an unlimited number of fields (within hardware limits).
  \item Tuple access is similar to calls, but there must only be one argument and it must be an integer literal. This is checked statically.
  \item Tuples in \textit{SEAL} are heterogenous, as is usual in most programming languages.
\end{itemize}

\subsection{Higher Order Functions}

\begin{lstlisting}
object HigherOrderFunctions {
  def map(f: Int $\Arr$ Int, xs: L.List): L.List = {
      xs match {
          case L.Nil() $\Arr$ L.Nil()
          case L.Cons(h, t) $\Arr$ L.Cons(f(h), map(f, t))
      }
  }
  
  def add(a: Int): Int $\Arr$ Int = {
      \(b: Int) $\arr$ a + b
  }
  
  val i: Int = 2;
  val xs0: L.List =
    L.Cons(1, L.Cons(2, L.Cons(3, L.Nil())));
  val xs1: L.List = map(\(x: Int) $\arr$ x + 1, xs0);
  val xs2: L.List = map(add(10), xs0);
  Std.printString(L.toString(xs1)); // List(2, 3, 4).
  Std.printString(L.toString(xs2)); // List(11, 12, 13).

  // \(x: Int) $\arr$ val y: Int = 1; x + y; // Error.
  \(x: Int) $\arr$ (val y: Int = 1; x + y); // OK.

  (\(x: Int): Int $\arr$ x + 1)(1); // Return type optional.

  val f: Int $\Arr$ Int $\Arr$ Int = add;
  Std.printInt(add(40)(2)); // Currying.
}
\end{lstlisting}
See \texttt{extension-examples/HigherOrder\\Functions.scala} for a fuller demonstration.

Functions are now supreme beings and have earned their rightful place as first-class citizens. The following features are included in the extension:

\begin{itemize}
  \item Function types: \code{A $\Arr$ B}
  \item Lambdas (anonymous functions): \\
    \code{\bs(x: Int) $\arr$ x + 1}
  \item Currying: \code{f(x)(y)}
\end{itemize}

\noindent
The syntax and semantics are as follows:

\begin{enumerate}[label=(\alph*)]
  \item Function types
  \begin{enumerate}[label=(\roman*)]
    \item Multiple parameters are specified as tuples, e.g. \code{(Int, Boolean, String) $\Arr$ Unit} represents a function of three arguments mapping to \code{Unit}.
    \item The type \code{()} is a synonym for \code{Unit}, e.g. \code{() $\Arr$ ()} and \code{Unit $\Arr$ Unit} are the same type.
    \item Similar to Scala, \code{((A, B)) $\Arr$ C} represents a function of one argument (here, a 2-tuple) mapping to \code{C}. This is to avoid confusion with \code{(A, B) $\Arr$ C}, which represents a function of \textit{two} arguments of type \code{A}, \code{B}, mapping to \code{C}.

    Note that the extra parentheses here is for the special case of functions with one tuple argument. For any other type of argument, the parentheses are superfluous (e.g. \code{(Int) $\Arr$ String} is the same as \code{Int $\Arr$ String}).
    \item Unlike Scala, it is invalid to cast a function from \code{(A1, $\dots$, An) $\Arr$ B} to \code{((A1, $\dots$, An)) $\Arr$ B} (i.e. from a function of $n$ arguments to a function of one $n$-tuple argument).
    \item Also unlike Scala, it is invalid to call \code{f(a1, $\dots$, an)} where \code{f} has type \code{((A1, $\dots$, An)) $\Arr$ B} and \code{ai} has type \code{Ai}. In other words, parameters won't be auto-squished into a tuple.
  \end{enumerate}
  \item Lambdas
  \begin{enumerate}[label=(\roman*)]
    \item Any free variables in a lambda (i.e. variables that reference names outside the lambda) are captured when the lambda is instantiated. Due to the immutable nature of Amy/\textit{SEAL}, variables are captured by value, so that there is no cause for alarm about dangling references.
    \item The return type of lambdas can be optionally specified: \code{\bs(x: Int): Int $\arr$ x + 1}. If no return type is specified for a lambda, it will be automatically deduced.
    \item The body of a lambda is not allowed to begin with a let-expression. Should there be a need for declaring local variables within a lambda, use a subexpression.
  \end{enumerate}
  \item Calls and currying
  \begin{enumerate}[label=(\roman*)]
    \item Any number of calls may follow a valid identifier or parenthesised value.
    \item Calls in \textit{SEAL} and Amy have the same precedence.
  \end{enumerate}
\end{enumerate}
